% Options for packages loaded elsewhere
\PassOptionsToPackage{unicode}{hyperref}
\PassOptionsToPackage{hyphens}{url}
%
\documentclass[
]{article}
\usepackage{lmodern}
\usepackage{amssymb,amsmath}
\usepackage{ifxetex,ifluatex}
\ifnum 0\ifxetex 1\fi\ifluatex 1\fi=0 % if pdftex
  \usepackage[T1]{fontenc}
  \usepackage[utf8]{inputenc}
  \usepackage{textcomp} % provide euro and other symbols
\else % if luatex or xetex
  \usepackage{unicode-math}
  \defaultfontfeatures{Scale=MatchLowercase}
  \defaultfontfeatures[\rmfamily]{Ligatures=TeX,Scale=1}
\fi
% Use upquote if available, for straight quotes in verbatim environments
\IfFileExists{upquote.sty}{\usepackage{upquote}}{}
\IfFileExists{microtype.sty}{% use microtype if available
  \usepackage[]{microtype}
  \UseMicrotypeSet[protrusion]{basicmath} % disable protrusion for tt fonts
}{}
\makeatletter
\@ifundefined{KOMAClassName}{% if non-KOMA class
  \IfFileExists{parskip.sty}{%
    \usepackage{parskip}
  }{% else
    \setlength{\parindent}{0pt}
    \setlength{\parskip}{6pt plus 2pt minus 1pt}}
}{% if KOMA class
  \KOMAoptions{parskip=half}}
\makeatother
\usepackage{xcolor}
\IfFileExists{xurl.sty}{\usepackage{xurl}}{} % add URL line breaks if available
\IfFileExists{bookmark.sty}{\usepackage{bookmark}}{\usepackage{hyperref}}
\hypersetup{
  pdftitle={Assignment 5: Data Visualization},
  pdfauthor={Sierra Kindley},
  hidelinks,
  pdfcreator={LaTeX via pandoc}}
\urlstyle{same} % disable monospaced font for URLs
\usepackage[margin=2.54cm]{geometry}
\usepackage{color}
\usepackage{fancyvrb}
\newcommand{\VerbBar}{|}
\newcommand{\VERB}{\Verb[commandchars=\\\{\}]}
\DefineVerbatimEnvironment{Highlighting}{Verbatim}{commandchars=\\\{\}}
% Add ',fontsize=\small' for more characters per line
\usepackage{framed}
\definecolor{shadecolor}{RGB}{248,248,248}
\newenvironment{Shaded}{\begin{snugshade}}{\end{snugshade}}
\newcommand{\AlertTok}[1]{\textcolor[rgb]{0.94,0.16,0.16}{#1}}
\newcommand{\AnnotationTok}[1]{\textcolor[rgb]{0.56,0.35,0.01}{\textbf{\textit{#1}}}}
\newcommand{\AttributeTok}[1]{\textcolor[rgb]{0.77,0.63,0.00}{#1}}
\newcommand{\BaseNTok}[1]{\textcolor[rgb]{0.00,0.00,0.81}{#1}}
\newcommand{\BuiltInTok}[1]{#1}
\newcommand{\CharTok}[1]{\textcolor[rgb]{0.31,0.60,0.02}{#1}}
\newcommand{\CommentTok}[1]{\textcolor[rgb]{0.56,0.35,0.01}{\textit{#1}}}
\newcommand{\CommentVarTok}[1]{\textcolor[rgb]{0.56,0.35,0.01}{\textbf{\textit{#1}}}}
\newcommand{\ConstantTok}[1]{\textcolor[rgb]{0.00,0.00,0.00}{#1}}
\newcommand{\ControlFlowTok}[1]{\textcolor[rgb]{0.13,0.29,0.53}{\textbf{#1}}}
\newcommand{\DataTypeTok}[1]{\textcolor[rgb]{0.13,0.29,0.53}{#1}}
\newcommand{\DecValTok}[1]{\textcolor[rgb]{0.00,0.00,0.81}{#1}}
\newcommand{\DocumentationTok}[1]{\textcolor[rgb]{0.56,0.35,0.01}{\textbf{\textit{#1}}}}
\newcommand{\ErrorTok}[1]{\textcolor[rgb]{0.64,0.00,0.00}{\textbf{#1}}}
\newcommand{\ExtensionTok}[1]{#1}
\newcommand{\FloatTok}[1]{\textcolor[rgb]{0.00,0.00,0.81}{#1}}
\newcommand{\FunctionTok}[1]{\textcolor[rgb]{0.00,0.00,0.00}{#1}}
\newcommand{\ImportTok}[1]{#1}
\newcommand{\InformationTok}[1]{\textcolor[rgb]{0.56,0.35,0.01}{\textbf{\textit{#1}}}}
\newcommand{\KeywordTok}[1]{\textcolor[rgb]{0.13,0.29,0.53}{\textbf{#1}}}
\newcommand{\NormalTok}[1]{#1}
\newcommand{\OperatorTok}[1]{\textcolor[rgb]{0.81,0.36,0.00}{\textbf{#1}}}
\newcommand{\OtherTok}[1]{\textcolor[rgb]{0.56,0.35,0.01}{#1}}
\newcommand{\PreprocessorTok}[1]{\textcolor[rgb]{0.56,0.35,0.01}{\textit{#1}}}
\newcommand{\RegionMarkerTok}[1]{#1}
\newcommand{\SpecialCharTok}[1]{\textcolor[rgb]{0.00,0.00,0.00}{#1}}
\newcommand{\SpecialStringTok}[1]{\textcolor[rgb]{0.31,0.60,0.02}{#1}}
\newcommand{\StringTok}[1]{\textcolor[rgb]{0.31,0.60,0.02}{#1}}
\newcommand{\VariableTok}[1]{\textcolor[rgb]{0.00,0.00,0.00}{#1}}
\newcommand{\VerbatimStringTok}[1]{\textcolor[rgb]{0.31,0.60,0.02}{#1}}
\newcommand{\WarningTok}[1]{\textcolor[rgb]{0.56,0.35,0.01}{\textbf{\textit{#1}}}}
\usepackage{graphicx,grffile}
\makeatletter
\def\maxwidth{\ifdim\Gin@nat@width>\linewidth\linewidth\else\Gin@nat@width\fi}
\def\maxheight{\ifdim\Gin@nat@height>\textheight\textheight\else\Gin@nat@height\fi}
\makeatother
% Scale images if necessary, so that they will not overflow the page
% margins by default, and it is still possible to overwrite the defaults
% using explicit options in \includegraphics[width, height, ...]{}
\setkeys{Gin}{width=\maxwidth,height=\maxheight,keepaspectratio}
% Set default figure placement to htbp
\makeatletter
\def\fps@figure{htbp}
\makeatother
\setlength{\emergencystretch}{3em} % prevent overfull lines
\providecommand{\tightlist}{%
  \setlength{\itemsep}{0pt}\setlength{\parskip}{0pt}}
\setcounter{secnumdepth}{-\maxdimen} % remove section numbering

\title{Assignment 5: Data Visualization}
\author{Sierra Kindley}
\date{}

\begin{document}
\maketitle

\hypertarget{overview}{%
\subsection{OVERVIEW}\label{overview}}

This exercise accompanies the lessons in Environmental Data Analytics on
Data Visualization

\hypertarget{directions}{%
\subsection{Directions}\label{directions}}

\begin{enumerate}
\def\labelenumi{\arabic{enumi}.}
\tightlist
\item
  Change ``Student Name'' on line 3 (above) with your name.
\item
  Work through the steps, \textbf{creating code and output} that fulfill
  each instruction.
\item
  Be sure to \textbf{answer the questions} in this assignment document.
\item
  When you have completed the assignment, \textbf{Knit} the text and
  code into a single PDF file.
\item
  After Knitting, submit the completed exercise (PDF file) to the
  dropbox in Sakai. Add your last name into the file name (e.g.,
  ``Salk\_A05\_DataVisualization.Rmd'') prior to submission.
\end{enumerate}

The completed exercise is due on Tuesday, February 11 at 1:00 pm.

\hypertarget{set-up-your-session}{%
\subsection{Set up your session}\label{set-up-your-session}}

\begin{enumerate}
\def\labelenumi{\arabic{enumi}.}
\item
  Set up your session. Verify your working directory and load the
  tidyverse and cowplot packages. Upload the NTL-LTER processed data
  files for nutrients and chemistry/physics for Peter and Paul Lakes
  (tidy and gathered) and the processed data file for the Niwot Ridge
  litter dataset.
\item
  Make sure R is reading dates as date format; if not change the format
  to date.
\end{enumerate}

\begin{Shaded}
\begin{Highlighting}[]
\CommentTok{#1}
\KeywordTok{getwd}\NormalTok{() }\CommentTok{#verify working directory}
\end{Highlighting}
\end{Shaded}

\begin{verbatim}
## [1] "C:/Users/sierr/Documents/Duke University/ENVIRON-872L/Environmental_Data_Analytics_2020"
\end{verbatim}

\begin{Shaded}
\begin{Highlighting}[]
\KeywordTok{library}\NormalTok{(tidyverse) }\CommentTok{#load tidyverse packages}
\end{Highlighting}
\end{Shaded}

\begin{verbatim}
## -- Attaching packages ------------------------------------------------------------------------ tidyverse 1.3.0 --
\end{verbatim}

\begin{verbatim}
## v ggplot2 3.2.1     v purrr   0.3.3
## v tibble  2.1.3     v dplyr   0.8.3
## v tidyr   1.0.0     v stringr 1.4.0
## v readr   1.3.1     v forcats 0.4.0
\end{verbatim}

\begin{verbatim}
## -- Conflicts --------------------------------------------------------------------------- tidyverse_conflicts() --
## x dplyr::filter() masks stats::filter()
## x dplyr::lag()    masks stats::lag()
\end{verbatim}

\begin{Shaded}
\begin{Highlighting}[]
\KeywordTok{library}\NormalTok{(cowplot) }\CommentTok{#load cowplot package}
\end{Highlighting}
\end{Shaded}

\begin{verbatim}
## 
## ********************************************************
\end{verbatim}

\begin{verbatim}
## Note: As of version 1.0.0, cowplot does not change the
\end{verbatim}

\begin{verbatim}
##   default ggplot2 theme anymore. To recover the previous
\end{verbatim}

\begin{verbatim}
##   behavior, execute:
##   theme_set(theme_cowplot())
\end{verbatim}

\begin{verbatim}
## ********************************************************
\end{verbatim}

\begin{Shaded}
\begin{Highlighting}[]
\NormalTok{PeterPaul.chem.nutrients <-}\StringTok{ }
\StringTok{  }\KeywordTok{read.csv}\NormalTok{(}\StringTok{"./Data/Processed/NTL-LTER_Lake_Chemistry_Nutrients_PeterPaul_Processed.csv"}\NormalTok{)}
\NormalTok{PeterPaul.chem.nutrients.gathered <-}\StringTok{ }
\StringTok{  }\KeywordTok{read.csv}\NormalTok{(}\StringTok{"./Data/Processed/NTL-LTER_Lake_Nutrients_PeterPaulGathered_Processed.csv"}\NormalTok{)}
\CommentTok{#upload NTL-LTER processed data files }
\NormalTok{NiwotRidge.litter <-}\StringTok{ }
\StringTok{  }\KeywordTok{read.csv}\NormalTok{(}\StringTok{"./Data/Processed/NEON_NIWO_Litter_mass_trap_Processed.csv"}\NormalTok{)}
\CommentTok{#upload Niwot Ridge litter processed data file}

\CommentTok{#2}
\KeywordTok{class}\NormalTok{(PeterPaul.chem.nutrients}\OperatorTok{$}\NormalTok{sampledate) }\CommentTok{#check class of 'sampledate' column}
\end{Highlighting}
\end{Shaded}

\begin{verbatim}
## [1] "factor"
\end{verbatim}

\begin{Shaded}
\begin{Highlighting}[]
\NormalTok{PeterPaul.chem.nutrients}\OperatorTok{$}\NormalTok{sampledate <-}\StringTok{ }
\StringTok{  }\KeywordTok{as.Date}\NormalTok{(PeterPaul.chem.nutrients}\OperatorTok{$}\NormalTok{sampledate, }\DataTypeTok{format =} \StringTok{"%Y-%m-%d"}\NormalTok{) }
  \CommentTok{#format 'sampledate' column as date}

\KeywordTok{class}\NormalTok{(PeterPaul.chem.nutrients.gathered}\OperatorTok{$}\NormalTok{sampledate) }\CommentTok{#check class of 'sampledate' column}
\end{Highlighting}
\end{Shaded}

\begin{verbatim}
## [1] "factor"
\end{verbatim}

\begin{Shaded}
\begin{Highlighting}[]
\NormalTok{PeterPaul.chem.nutrients.gathered}\OperatorTok{$}\NormalTok{sampledate <-}\StringTok{ }
\StringTok{  }\KeywordTok{as.Date}\NormalTok{(PeterPaul.chem.nutrients.gathered}\OperatorTok{$}\NormalTok{sampledate, }\DataTypeTok{format =} \StringTok{"%Y-%m-%d"}\NormalTok{) }
  \CommentTok{#format 'sampledate' column as date}

\KeywordTok{class}\NormalTok{(NiwotRidge.litter}\OperatorTok{$}\NormalTok{collectDate) }\CommentTok{#check class of 'collectDate' column}
\end{Highlighting}
\end{Shaded}

\begin{verbatim}
## [1] "factor"
\end{verbatim}

\begin{Shaded}
\begin{Highlighting}[]
\NormalTok{NiwotRidge.litter}\OperatorTok{$}\NormalTok{collectDate <-}\StringTok{ }
\StringTok{  }\KeywordTok{as.Date}\NormalTok{(NiwotRidge.litter}\OperatorTok{$}\NormalTok{collectDate, }\DataTypeTok{format =} \StringTok{"%Y-%m-%d"}\NormalTok{) }
  \CommentTok{#format 'collectDate' column as date}
\end{Highlighting}
\end{Shaded}

\hypertarget{define-your-theme}{%
\subsection{Define your theme}\label{define-your-theme}}

\begin{enumerate}
\def\labelenumi{\arabic{enumi}.}
\setcounter{enumi}{2}
\tightlist
\item
  Build a theme and set it as your default theme.
\end{enumerate}

\begin{Shaded}
\begin{Highlighting}[]
\NormalTok{sierratheme <-}\StringTok{ }\KeywordTok{theme_classic}\NormalTok{(}\DataTypeTok{base_size =} \DecValTok{14}\NormalTok{) }\OperatorTok{+}
\StringTok{  }\KeywordTok{theme}\NormalTok{(}\DataTypeTok{axis.text =} \KeywordTok{element_text}\NormalTok{(}\DataTypeTok{color =} \StringTok{"black"}\NormalTok{), }
        \DataTypeTok{legend.position =} \StringTok{"top"}\NormalTok{)}
\CommentTok{#build new theme 'sierratheme' and define what its parameters}
\KeywordTok{theme_set}\NormalTok{(sierratheme) }\CommentTok{#set sierratheme as default theme}
\end{Highlighting}
\end{Shaded}

\hypertarget{create-graphs}{%
\subsection{Create graphs}\label{create-graphs}}

For numbers 4-7, create ggplot graphs and adjust aesthetics to follow
best practices for data visualization. Ensure your theme, color
palettes, axes, and additional aesthetics are edited accordingly.

\begin{enumerate}
\def\labelenumi{\arabic{enumi}.}
\setcounter{enumi}{3}
\tightlist
\item
  {[}NTL-LTER{]} Plot total phosphorus by phosphate, with separate
  aesthetics for Peter and Paul lakes. Add a line of best fit and color
  it black. Adjust your axes to hide extreme values.
\end{enumerate}

\begin{Shaded}
\begin{Highlighting}[]
\NormalTok{PeterPaulPlot1 <-}\StringTok{ }\KeywordTok{ggplot}\NormalTok{(PeterPaul.chem.nutrients, }
                          \KeywordTok{aes}\NormalTok{(}\DataTypeTok{x =}\NormalTok{ tp_ug, }\DataTypeTok{y =}\NormalTok{ po4, }\DataTypeTok{color =}\NormalTok{ lakename)) }\OperatorTok{+}
\StringTok{  }\KeywordTok{geom_point}\NormalTok{() }\OperatorTok{+}
\StringTok{  }\KeywordTok{ylim}\NormalTok{(}\DecValTok{0}\NormalTok{,}\DecValTok{30}\NormalTok{) }\OperatorTok{+}\StringTok{ }\CommentTok{#define limits of y-axis}
\StringTok{  }\KeywordTok{xlim}\NormalTok{(}\DecValTok{0}\NormalTok{,}\DecValTok{150}\NormalTok{) }\OperatorTok{+}\StringTok{ }\CommentTok{#define limits of x-axis}
\StringTok{  }\KeywordTok{ylab}\NormalTok{(}\KeywordTok{expression}\NormalTok{(}\KeywordTok{paste}\NormalTok{(}\StringTok{"PO"}\NormalTok{[}\DecValTok{4}\NormalTok{]}\OperatorTok{*}\StringTok{ " ("}\NormalTok{, mu, }\StringTok{"g/L)"}\NormalTok{))) }\OperatorTok{+}\StringTok{ }\CommentTok{#change y-label}
\StringTok{  }\KeywordTok{xlab}\NormalTok{(}\KeywordTok{expression}\NormalTok{(}\KeywordTok{paste}\NormalTok{(}\StringTok{"TP ("}\NormalTok{, mu, }\StringTok{"g/L)"}\NormalTok{))) }\OperatorTok{+}\StringTok{ }\CommentTok{#change x-label}
\StringTok{  }\KeywordTok{geom_smooth}\NormalTok{(}\DataTypeTok{method =}\NormalTok{ lm, }\DataTypeTok{se =} \OtherTok{FALSE}\NormalTok{, }\DataTypeTok{color =} \StringTok{"black"}\NormalTok{) }\OperatorTok{+}\StringTok{ }\CommentTok{#add black line of best fit }
\StringTok{  }\KeywordTok{scale_color_manual}\NormalTok{(}\DataTypeTok{values =} \KeywordTok{c}\NormalTok{(}\StringTok{"#0c2c84"}\NormalTok{, }\StringTok{"#7fcdbb"}\NormalTok{)) }\OperatorTok{+}\StringTok{ }\CommentTok{#assign colors to Peter & Paul Lakes}
\StringTok{  }\KeywordTok{theme}\NormalTok{(}\DataTypeTok{legend.title =} \KeywordTok{element_blank}\NormalTok{()) }\CommentTok{#remove legend title}
\KeywordTok{print}\NormalTok{(PeterPaulPlot1) }\CommentTok{#print plot }
\end{Highlighting}
\end{Shaded}

\begin{verbatim}
## Warning: Removed 21955 rows containing non-finite values (stat_smooth).
\end{verbatim}

\begin{verbatim}
## Warning: Removed 21955 rows containing missing values (geom_point).
\end{verbatim}

\includegraphics{A05_DataVisualization_files/figure-latex/unnamed-chunk-3-1.pdf}

\begin{enumerate}
\def\labelenumi{\arabic{enumi}.}
\setcounter{enumi}{4}
\tightlist
\item
  {[}NTL-LTER{]} Make three separate boxplots of (a) temperature, (b)
  TP, and (c) TN, with month as the x axis and lake as a color
  aesthetic. Then, create a cowplot that combines the three graphs. Make
  sure that only one legend is present and that graph axes are aligned.
\end{enumerate}

\begin{Shaded}
\begin{Highlighting}[]
\NormalTok{PeterPaulPlot2 <-}\StringTok{ }\KeywordTok{ggplot}\NormalTok{(}\KeywordTok{subset}\NormalTok{(PeterPaul.chem.nutrients, depth }\OperatorTok{<}\StringTok{ }\FloatTok{0.25}\NormalTok{)) }\OperatorTok{+}
\StringTok{  }\KeywordTok{geom_boxplot}\NormalTok{(}\KeywordTok{aes}\NormalTok{(}\DataTypeTok{x =} \KeywordTok{as.factor}\NormalTok{(month), }\DataTypeTok{y =}\NormalTok{ temperature_C, }\DataTypeTok{color =}\NormalTok{ lakename)) }\OperatorTok{+}
\StringTok{  }\KeywordTok{xlab}\NormalTok{(}\KeywordTok{expression}\NormalTok{(}\KeywordTok{paste}\NormalTok{(}\StringTok{"Month"}\NormalTok{))) }\OperatorTok{+}\StringTok{ }\CommentTok{#change x-label}
\StringTok{  }\KeywordTok{ylab}\NormalTok{(}\KeywordTok{expression}\NormalTok{(}\KeywordTok{paste}\NormalTok{(}\StringTok{"Temperature ("}\OperatorTok{*}\ErrorTok{~}\NormalTok{degree}\OperatorTok{*}\NormalTok{C}\OperatorTok{*}\StringTok{")"}\NormalTok{)))}\OperatorTok{+}\StringTok{ }\CommentTok{#change y-label}
\StringTok{  }\KeywordTok{scale_color_manual}\NormalTok{(}\DataTypeTok{values =} \KeywordTok{c}\NormalTok{(}\StringTok{"#0c2c84"}\NormalTok{, }\StringTok{"#7fcdbb"}\NormalTok{)) }\OperatorTok{+}\StringTok{ }\CommentTok{#assign colors to Peter & Paul Lakes}
\StringTok{  }\KeywordTok{theme}\NormalTok{(}\DataTypeTok{legend.title =} \KeywordTok{element_blank}\NormalTok{()) }\CommentTok{#remove legend title}
  
\NormalTok{PeterPaulPlot3 <-}\StringTok{ }\KeywordTok{ggplot}\NormalTok{(}\KeywordTok{subset}\NormalTok{(PeterPaul.chem.nutrients, depth }\OperatorTok{<}\StringTok{ }\FloatTok{0.25}\NormalTok{)) }\OperatorTok{+}
\StringTok{  }\KeywordTok{geom_boxplot}\NormalTok{(}\KeywordTok{aes}\NormalTok{(}\DataTypeTok{x =} \KeywordTok{as.factor}\NormalTok{(month), }\DataTypeTok{y =}\NormalTok{ tp_ug, }\DataTypeTok{color =}\NormalTok{ lakename)) }\OperatorTok{+}
\StringTok{  }\KeywordTok{xlab}\NormalTok{(}\KeywordTok{expression}\NormalTok{(}\KeywordTok{paste}\NormalTok{(}\StringTok{"Month"}\NormalTok{))) }\OperatorTok{+}\StringTok{ }\CommentTok{#change x-label}
\StringTok{  }\KeywordTok{ylab}\NormalTok{(}\KeywordTok{expression}\NormalTok{(}\KeywordTok{paste}\NormalTok{(}\StringTok{"TP ("}\NormalTok{, mu, }\StringTok{"g/L)"}\NormalTok{))) }\OperatorTok{+}\StringTok{ }\CommentTok{#change y-label}
\StringTok{  }\KeywordTok{ylim}\NormalTok{(}\DecValTok{0}\NormalTok{,}\DecValTok{100}\NormalTok{) }\OperatorTok{+}\StringTok{ }\CommentTok{#define limits of y-axis}
\StringTok{  }\KeywordTok{scale_color_manual}\NormalTok{(}\DataTypeTok{values =} \KeywordTok{c}\NormalTok{(}\StringTok{"#0c2c84"}\NormalTok{, }\StringTok{"#7fcdbb"}\NormalTok{)) }\OperatorTok{+}\StringTok{ }\CommentTok{#assign colors to Peter & Paul Lakes}
\StringTok{  }\KeywordTok{theme}\NormalTok{(}\DataTypeTok{legend.position =} \StringTok{"none"}\NormalTok{) }\CommentTok{#remove legend from plot}

\NormalTok{PeterPaulPlot4 <-}\StringTok{ }\KeywordTok{ggplot}\NormalTok{(}\KeywordTok{subset}\NormalTok{(PeterPaul.chem.nutrients, depth }\OperatorTok{<}\StringTok{ }\FloatTok{0.25}\NormalTok{)) }\OperatorTok{+}\StringTok{ }
\StringTok{  }\KeywordTok{geom_boxplot}\NormalTok{(}\KeywordTok{aes}\NormalTok{(}\DataTypeTok{x =} \KeywordTok{as.factor}\NormalTok{(month), }\DataTypeTok{y =}\NormalTok{ tn_ug, }\DataTypeTok{color =}\NormalTok{ lakename)) }\OperatorTok{+}
\StringTok{  }\KeywordTok{xlab}\NormalTok{(}\KeywordTok{expression}\NormalTok{(}\KeywordTok{paste}\NormalTok{(}\StringTok{"Month"}\NormalTok{))) }\OperatorTok{+}\StringTok{ }\CommentTok{#change x-label}
\StringTok{  }\KeywordTok{ylab}\NormalTok{(}\KeywordTok{expression}\NormalTok{(}\KeywordTok{paste}\NormalTok{(}\StringTok{"TN ("}\NormalTok{, mu, }\StringTok{"g/L)"}\NormalTok{))) }\OperatorTok{+}\StringTok{ }\CommentTok{#change y-label}
\StringTok{  }\KeywordTok{ylim}\NormalTok{(}\DecValTok{0}\NormalTok{,}\DecValTok{2300}\NormalTok{) }\OperatorTok{+}\StringTok{ }\CommentTok{#define limits of y-axis}
\StringTok{  }\KeywordTok{scale_color_manual}\NormalTok{(}\DataTypeTok{values =} \KeywordTok{c}\NormalTok{(}\StringTok{"#0c2c84"}\NormalTok{, }\StringTok{"#7fcdbb"}\NormalTok{)) }\OperatorTok{+}\StringTok{ }\CommentTok{#assign colors to Peter & Paul Lakes}
\StringTok{  }\KeywordTok{theme}\NormalTok{(}\DataTypeTok{legend.position =} \StringTok{"none"}\NormalTok{) }\CommentTok{#remove legend from plot}

\KeywordTok{plot_grid}\NormalTok{(PeterPaulPlot2, PeterPaulPlot3, PeterPaulPlot4, }\DataTypeTok{nrow =} \DecValTok{3}\NormalTok{, }
          \DataTypeTok{align =} \StringTok{'v'}\NormalTok{, }\DataTypeTok{rel_heights =} \KeywordTok{c}\NormalTok{(}\FloatTok{1.25}\NormalTok{, }\DecValTok{1}\NormalTok{, }\DecValTok{1}\NormalTok{))}
\end{Highlighting}
\end{Shaded}

\begin{verbatim}
## Warning: Removed 17 rows containing non-finite values (stat_boxplot).
\end{verbatim}

\begin{verbatim}
## Warning: Removed 806 rows containing non-finite values (stat_boxplot).
\end{verbatim}

\begin{verbatim}
## Warning: Removed 872 rows containing non-finite values (stat_boxplot).
\end{verbatim}

\includegraphics{A05_DataVisualization_files/figure-latex/unnamed-chunk-4-1.pdf}

\begin{Shaded}
\begin{Highlighting}[]
\CommentTok{#combine three previous graphs into one cowplot with one legend and aligned axes}
\end{Highlighting}
\end{Shaded}

Question: What do you observe about the variables of interest over
seasons and between lakes?

\begin{quote}
Answer: The temperature in the lakes appears slightly higher in the
summer months and, in general, the temperature does not vary too much
between lakes. Total phosphorous does not appear to vary much over
seasons, but the TP values in Peter Lake are consistently higher than
those in Paul Lake. Similarly, total nitrogen values do not vary much
over seasons, but the TP values in Peter Lake are consistently higher
than those in Paul Lake.
\end{quote}

\begin{enumerate}
\def\labelenumi{\arabic{enumi}.}
\setcounter{enumi}{5}
\item
  {[}Niwot Ridge{]} Plot a subset of the litter dataset by displaying
  only the ``Needles'' functional group. Plot the dry mass of needle
  litter by date and separate by NLCD class with a color aesthetic. (no
  need to adjust the name of each land use)
\item
  {[}Niwot Ridge{]} Now, plot the same plot but with NLCD classes
  separated into three facets rather than separated by color.
\end{enumerate}

\begin{Shaded}
\begin{Highlighting}[]
\CommentTok{#6}
\NormalTok{NiwotRidgePlot1 <-}\StringTok{ }
\StringTok{  }\KeywordTok{ggplot}\NormalTok{(}\KeywordTok{subset}\NormalTok{(NiwotRidge.litter, functionalGroup }\OperatorTok{==}\StringTok{ "Needles"}\NormalTok{),}
                \KeywordTok{aes}\NormalTok{(}\DataTypeTok{x =}\NormalTok{ collectDate, }\DataTypeTok{y =}\NormalTok{ dryMass, }\DataTypeTok{color =}\NormalTok{ nlcdClass)) }\OperatorTok{+}
\StringTok{  }\KeywordTok{geom_point}\NormalTok{() }\OperatorTok{+}
\StringTok{  }\KeywordTok{xlab}\NormalTok{(}\KeywordTok{expression}\NormalTok{(}\KeywordTok{paste}\NormalTok{(}\StringTok{"Date"}\NormalTok{))) }\OperatorTok{+}\StringTok{ }\CommentTok{#change x-label}
\StringTok{  }\KeywordTok{ylab}\NormalTok{(}\KeywordTok{expression}\NormalTok{(}\KeywordTok{paste}\NormalTok{(}\StringTok{"Dry Mass (g)"}\NormalTok{))) }\OperatorTok{+}\StringTok{ }\CommentTok{#change y-label}
\StringTok{  }\KeywordTok{scale_color_manual}\NormalTok{(}\DataTypeTok{values =} \KeywordTok{c}\NormalTok{(}\StringTok{"#7fcdbb"}\NormalTok{, }\StringTok{"#1d91c0"}\NormalTok{, }\StringTok{"#0c2c84"}\NormalTok{), }
                     \DataTypeTok{labels =} \KeywordTok{c}\NormalTok{(}\KeywordTok{expression}\NormalTok{(}\StringTok{"Evergreen Forest"}\NormalTok{), }
                                \KeywordTok{expression}\NormalTok{(}\StringTok{"Grassland Herbaceous"}\NormalTok{), }
                                \KeywordTok{expression}\NormalTok{(}\StringTok{"Shrub Scrub"}\NormalTok{))) }\OperatorTok{+}\StringTok{ }
\StringTok{  }\CommentTok{#assign colors & labels to data by NLCD class }
\StringTok{  }\KeywordTok{labs}\NormalTok{(}\DataTypeTok{color =} \StringTok{"Land Cover"}\NormalTok{)}
\KeywordTok{print}\NormalTok{(NiwotRidgePlot1) }\CommentTok{#print plot}
\end{Highlighting}
\end{Shaded}

\includegraphics{A05_DataVisualization_files/figure-latex/unnamed-chunk-5-1.pdf}

\begin{Shaded}
\begin{Highlighting}[]
\CommentTok{#7}
\NormalTok{NiwotRidgePlot1.faceted <-}\StringTok{ }
\StringTok{  }\KeywordTok{ggplot}\NormalTok{(}\KeywordTok{subset}\NormalTok{(NiwotRidge.litter, functionalGroup }\OperatorTok{==}\StringTok{ "Needles"}\NormalTok{)) }\OperatorTok{+}
\StringTok{  }\KeywordTok{geom_point}\NormalTok{(}\KeywordTok{aes}\NormalTok{(}\DataTypeTok{x =}\NormalTok{ collectDate, }\DataTypeTok{y =}\NormalTok{ dryMass)) }\OperatorTok{+}
\StringTok{  }\KeywordTok{facet_wrap}\NormalTok{(}\KeywordTok{vars}\NormalTok{(nlcdClass), }\DataTypeTok{nrow =} \DecValTok{3}\NormalTok{) }\OperatorTok{+}\StringTok{ }\CommentTok{#create facets by NLCD class}
\StringTok{  }\KeywordTok{xlab}\NormalTok{(}\KeywordTok{expression}\NormalTok{(}\KeywordTok{paste}\NormalTok{(}\StringTok{"Date"}\NormalTok{))) }\OperatorTok{+}\StringTok{ }\CommentTok{#change x-label}
\StringTok{  }\KeywordTok{ylab}\NormalTok{(}\KeywordTok{expression}\NormalTok{(}\KeywordTok{paste}\NormalTok{(}\StringTok{"Dry Mass (g)"}\NormalTok{))) }\OperatorTok{+}\StringTok{ }\CommentTok{#change y-label}
\StringTok{  }\KeywordTok{theme}\NormalTok{(}\DataTypeTok{legend.position =} \StringTok{"none"}\NormalTok{) }\CommentTok{#remove legend from plot}
\KeywordTok{print}\NormalTok{(NiwotRidgePlot1.faceted) }\CommentTok{#print plot}
\end{Highlighting}
\end{Shaded}

\includegraphics{A05_DataVisualization_files/figure-latex/unnamed-chunk-5-2.pdf}

Question: Which of these plots (6 vs.~7) do you think is more effective,
and why?

\begin{quote}
Answer: I think the faceted plot (plot in question 7) is more effective,
as it is more visually appealing and understandable. I think separating
the data into three facets based on NLCD class helps greatly with the
overall interpretation of the data, as it separates it out and allows
for clearer interpretation and easier comparison between NLCD classes.
The plot in question 6 is okay, but because all the data is lumped
together on a single plot, I find it much more difficult to visualize
and interpret the data, as well as distinguish and make comparisons
between the three NLCD classes.
\end{quote}

\end{document}
